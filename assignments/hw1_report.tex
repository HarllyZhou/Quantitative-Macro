\documentclass[12pt]{article}

\usepackage[a4paper,margin=1in]{geometry}
\usepackage{setspace}
\onehalfspacing

\usepackage{amsmath,amssymb,amsthm}
\usepackage{graphicx}
\usepackage{float}
\usepackage{caption}
\usepackage{subcaption}
\usepackage{enumitem}

\title{ECON 5345 Homework 1 Report}
\author{Harlly Zhou}
\date{\today}

\begin{document}

\maketitle

\section*{Question 1}
\begin{enumerate}[label=\alph*.]
    \item Note that for any $t$, we have
    \[
        C_t = C_{t-3} + e_{t-2} + e_{t-1} + e_t.
    \]
    Substituting this into the 
    \begin{align*}
        \Delta C_t &\equiv\frac{C_t + C_{t+1} + C_{t+2}}{3} - \frac{C_{t-3} + C_{t-2} + C_{t-1}}{3} \\
        &=
        \frac{e_{t-2} + 2e_{t-1} + 3e_t + 2e_{t+1} + e_{t+2}}{3}.
    \end{align*}

    \item No. They are correlated. At $t+3$, we have
    \begin{align*}
        \Delta C_{t+3} = \frac{e_{t+1} + 2e_{t+2} + 3e_{t+3} + 2e_{t+4} + e_{t+5}}{3}.
    \end{align*}
    It is clear that
    \begin{align*}
        \text{Cov}(\Delta C_t, \Delta C_{t+3}) = \frac{2}{9}(\text{Var}[e_{t+1}] + \text{Var}[e_{t+2}]) > 0,
    \end{align*}
    as long as $\text{Var}[e_{t+1}] + \text{Var}[e_{t+2}] > 0$.

    \item No for the first part. Since $e_{t-2}$ and $e_{t-1}$ are known, $\Delta C_t$ is correlated with $C_{t-2}$ and $C_{t-1}$. 
    
    Yes for the second part. Information known at $t-3$ only includes white noise no later then $t-3$, while $\Delta C_t$ is a linear combination of white noises after $t-3$. Given the serial uncorrelation property of white noise, they are not correlated.

    \item The ACF and PACF of the change in measured consumption are shown in Figure \ref{fig:1d}. Codes in ``hw1\_q1d.R''.
    \begin{figure}[htbp]
        \centering
        \includegraphics[width=0.6\textwidth]{hw1_q1d_acf_pacf.png}
        \caption{ACF and PACF of the change in measured consumption}
        \label{fig:1d}
    \end{figure}

    \item The ACF and PACF of the change in measured consumption are shown in Figure \ref{fig:1e_month} and Figure \ref{fig:1e_quarter}. Codes in ``hw1\_q1e.R''.
    \begin{figure}[htbp]
        \centering
        \begin{subfigure}[t]{0.6\textwidth}
            \centering
            \includegraphics[width=\textwidth]{hw1_q1e_month_acf_pacf.png}
            \caption{Monthly data: Jan 2007--Sep 2025 (exclude months after Sep 2025 since quarterly data ends at 2025Q3).}
            \label{fig:1e_month}
        \end{subfigure}

        \vspace{0.8em}

        \begin{subfigure}[t]{0.6\textwidth}
            \centering
            \includegraphics[width=\textwidth]{hw1_q1e_quarter_acf_pacf.png}
            \caption{Quarterly data: 2007Q1--2025Q3.}
            \label{fig:1e_quarter}
        \end{subfigure}
        \caption{ACF and PACF of the change in consumption (monthly and quarterly). Data source: FRED PCENDC96.}
        \label{fig:1e}
    \end{figure}

    Monthly data shows negative autocorrelation at lag 1, while quaterly data shows positive autocorrelation at lag 1 and 2, even 3. This shows higher persistence in quarterly data, reflecting the fact that averaging over more months introduces more serial correlation, as shown in part (b).
    
\end{enumerate}

\newpage


\section*{Question 2}
\begin{enumerate}[label=\alph*.]
    \item Since $d$ is observable, the firm has the following optimization problem:
    \begin{align*}
        \max_{p}\, \Pi(p,d) = p^{-\frac{d+1}{d}}(p-1).
    \end{align*}
    The FOC condition gives
    \begin{align*}
        -\frac{d+1}{d} p^{-\frac{d+1}{d}-1}(p-1) + p^{-\frac{d+1}{d}} = 0.
    \end{align*}
    Multiplying both sides by $p^{\frac{d+1}{d}}$ and rearranging the terms, we get
    \begin{align*}
        p^*(d) = 1 + d.
    \end{align*}

    \item The code is in ``hw1\_q2b.R''. Since I used R, something may be different from the Matlab code. To find the initial point, I used the help of AI because uniform initial point produces a flat plane.The 3D plot is shown in Figure \ref{fig:2b}.
    \begin{figure}[htbp]
        \centering
        \includegraphics[width=0.6\textwidth]{hw1_q2b_surface.png}
        \caption{Optimal joint density f(p, d)}
        \label{fig:2b}
    \end{figure}
\end{enumerate}

\newpage

\section*{Question 3}
\begin{enumerate}[label=\alph*.]
    \item No. When the observations are treated as cross-sectional data, the model becomes
    \begin{align*}
        y_{i1} = \mu_i + \rho y_{i0} + e_{i1} = \rho y_{i0} + \epsilon_{i1},
    \end{align*} 
    where $\epsilon_{i1} = \mu_i + e_{i1}$. However, note that, typically, we have
    \begin{align*}
        \text{Cov}(y_{i0}, \mu_i) &= \text{Cov}(\mu_i + \rho y_{i, -1} + e_{i0}, \mu_i) \\
        &= \text{Var}(\mu_i) + \rho \text{Cov}(y_{i, -1}, \mu_i) + \text{Cov}(e_{i0}, \mu_i) \\
        &\neq 0.
    \end{align*} 
    Then the orthogonality condition is violated since the error term also contains $\mu_i$.

    When setting $\mu_i = \mu$ for all $i$'s, the covariance becomes zero since $y_{i0}$ have zero covariance with constant $\mu$.

    The estimators will converge to the true value as $N \to \infty$ if the orthogonality condition holds.

    \item Yes. The model now becomes
    \begin{align*}
        y_t = \mu + \rho y_{t-1} + e_t.
    \end{align*}
    Using backward induction, we can write $y_t$ as a function of $\{e_s | s \leq t\}$, denoted by $h(\mathbf{e_t})$, where $\mathbf{e_t}$ denotes the vector of all $e_s$'s with $s\leq t$. Then independence between $e_s$'s implies that
    \begin{align*}
        \mathbb{E} [y_{t-1} e_t] = \mathbb{E} [h(\mathbf{e_{t-1}}) e_t] = 0.
    \end{align*}
    Since $e_s$ is with zero mean, the orthogonality conditions are satisfied and OLS works.

    \item 
    \begin{table}[htbp]
        \centering
        \caption{Q3(c): Fixed-effects (demeaned) OLS estimates with $\rho=0.9$, $\sigma=1$, $T=2$.}
        \label{tab:q3c}
        \begin{tabular}{rrr}
          \hline
          $N$ & $T$ & $\hat{\rho}$ \\
          \hline
          100 & 2 & -0.006711 \\
          500 & 2 & -0.182214 \\
          1000 & 2 & -0.038676 \\
          \hline
        \end{tabular}
      \end{table}
      
      \item 
      \begin{table}[htbp]
        \centering
        \caption{Q3(d): Fixed-effects (demeaned) OLS estimates with $\\rho=0.9$, $\\sigma=1$, $N=1000$.}
        \label{tab:q3d}
        \begin{tabular}{rrr}
          \hline
          $N$ & $T$ & $\hat{\rho}$ \\
          \hline
          1000 & 5 & 0.450082 \\
          1000 & 20 & 0.776592 \\
          \hline
        \end{tabular}
      \end{table}
\end{enumerate}


\end{document}
