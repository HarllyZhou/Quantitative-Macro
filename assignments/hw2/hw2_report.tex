\documentclass[12pt]{article}

\usepackage[a4paper,margin=1in]{geometry}
\usepackage{setspace}
\onehalfspacing

\usepackage{amsmath,amssymb,amsthm}
\usepackage{graphicx}
\usepackage{float}
\usepackage{caption}
\usepackage{subcaption}
\usepackage{enumitem}

% \begin{figure}[htbp]
%     \centering
%     \includegraphics[width=0.6\textwidth]{hw1_q2b_surface.png}
%     \caption{Optimal joint density f(p, d)}
%     \label{fig:2b}
% \end{figure}

\title{ECON 5345 Homework 2 Report}
\author{Harlly Zhou}
\date{\today}

\begin{document}

\maketitle

\section*{Question 1}
\begin{enumerate}[label=\alph*.]
   \item Q1
\end{enumerate}

\newpage


\section*{Question 2}
\begin{enumerate}[label=\alph*.]
    \item The impulse response function is shown in Figure \ref{fig:2a}.
    \begin{figure}[htbp]
        \centering
        \includegraphics[width=0.6\textwidth]{hw2_q2a_irf.png}
        \caption{IRF to unit shock in $e_0$}
        \label{fig:2a}
    \end{figure}
    
    Numerical results are shown in Table \ref{tab:q2a}.
    \begin{table}

        \caption{Impulse response}
        \label{tab:q2a}
        \centering
        \begin{tabular}[t]{rr}
        \hline \hline
        t & irf\\
        \hline
        0 & 1.0000\\
        1 & 1.2000\\
        2 & 0.9600\\
        3 & 0.7680\\
        4 & 0.6144\\
        \hline
        5 & 0.4915\\
        6 & 0.3932\\
        7 & 0.3146\\
        8 & 0.2517\\
        9 & 0.2013\\
        \hline
        10 & 0.1611\\
        11 & 0.1288\\
        12 & 0.1031\\
        13 & 0.0825\\
        14 & 0.0660\\
        \hline
        15 & 0.0528\\
        16 & 0.0422\\
        17 & 0.0338\\
        18 & 0.0270\\
        19 & 0.0216\\
        \hline
        20 & 0.0173\\
        21 & 0.0138\\
        22 & 0.0111\\
        23 & 0.0089\\
        24 & 0.0071\\
        \hline
        25 & 0.0057\\
        \end{tabular}
        \end{table}
\end{enumerate}

\newpage

\section*{Question 3}
\begin{enumerate}[label=\alph*.]
   \item Q3
\end{enumerate}

\newpage

\section*{Question 4}
\begin{enumerate}[label=\alph*.]
   \item The expectation is
   \begin{align*}
      \mathbb{E}[x_t] &= \mathbb{E}[\alpha] \cos t + \mathbb{E}[\beta] \sin t\\
      & = 0.
   \end{align*}
   The variance is
   \begin{align*}
      \operatorname{Var}[x_t] &= \operatorname{Var}[\alpha] \cos^2 t + \operatorname{Var}[\beta] \sin^2 t + 2 \operatorname{Cov}(\alpha, \beta) \cos t \sin t\\
      & = \cos^2 t + \sin^2 t\\
      & = 1.
   \end{align*}
   The second line is by the independence of $\alpha$ and $\beta$.

   \item No. Note that
   \begin{align*}
      \mathbb{E}[\alpha^2] &= \operatorname{Var}[\alpha] + \left(\mathbb{E}[\alpha]\right)^2\\
      &= 1\\
      \mathbb{E}[\beta^2] &= \operatorname{Var}[\beta] + \left(\mathbb{E}[\beta]\right)^2\\
      &= 1\\
      \mathbb{E}[\alpha \beta] &= \operatorname{Cov}(\alpha, \beta) + \mathbb{E}[\alpha] \mathbb{E}[\beta]\\
      &= 0.
   \end{align*}
   As a result, we have
   \begin{align*}
      \mathbb{E}[x_t x_{t-k}] &= \mathbb{E}[\alpha^2] \cos t \cos (t-k) + \mathbb{E}[\beta^2] \sin t \sin (t-k)\\
      &\qquad \quad + \mathbb{E}[\alpha \beta] [\cos t \sin (t-k) + \sin t \cos (t-k)]\\
      &= \cos t \cos (t-k) + \sin t \sin (t-k)\\
      &= \cos [t-(t-k)]\\
      &= \cos k.
   \end{align*}
\end{enumerate}


\end{document}
